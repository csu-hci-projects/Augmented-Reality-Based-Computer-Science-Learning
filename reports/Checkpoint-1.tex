% $Id: template.tex 11 2007-04-03 22:25:53Z jpeltier $

\documentclass{vgtc}                          % final (conference style)
%\documentclass[review]{vgtc}                 % review
%\documentclass[widereview]{vgtc}             % wide-spaced review
%\documentclass[preprint]{vgtc}               % preprint
%\documentclass[electronic]{vgtc}             % electronic version

%% Uncomment one of the lines above depending on where your paper is
%% in the conference process. ``review'' and ``widereview'' are for review
%% submission, ``preprint'' is for pre-publication, and the final version
%% doesn't use a specific qualifier. Further, ``electronic'' includes
%% hyperreferences for more convenient online viewing.

%% Please use one of the ``review'' options in combination with the
%% assigned online id (see below) ONLY if your paper uses a double blind
%% review process. Some conferences, like IEEE Vis and InfoVis, have NOT
%% in the past.

%% Figures should be in CMYK or Grey scale format, otherwise, colour 
%% shifting may occur during the printing process.

%% These few lines make a distinction between latex and pdflatex calls and they
%% bring in essential packages for graphics and font handling.
%% Note that due to the \DeclareGraphicsExtensions{} call it is no longer necessary
%% to provide the the path and extension of a graphics file:
%% \includegraphics{diamondrule} is completely sufficient.
%%
\ifpdf%                                % if we use pdflatex
  \pdfoutput=1\relax                   % create PDFs from pdfLaTeX
  \pdfcompresslevel=9                  % PDF Compression
  \pdfoptionpdfminorversion=7          % create PDF 1.7
  \ExecuteOptions{pdftex}
  \usepackage{graphicx}                % allow us to embed graphics files
  \DeclareGraphicsExtensions{.pdf,.png,.jpg,.jpeg} % for pdflatex we expect .pdf, .png, or .jpg files
\else%                                 % else we use pure latex
  \ExecuteOptions{dvips}
  \usepackage{graphicx}                % allow us to embed graphics files
  \DeclareGraphicsExtensions{.eps}     % for pure latex we expect eps files
\fi%

%% it is recomended to use ``\autoref{sec:bla}'' instead of ``Fig.~\ref{sec:bla}''
\graphicspath{{figures/}{pictures/}{images/}{./}} % where to search for the images

\usepackage{microtype}                 % use micro-typography (slightly more compact, better to read)
\PassOptionsToPackage{warn}{textcomp}  % to address font issues with \textrightarrow
\usepackage{textcomp}                  % use better special symbols
\usepackage{mathptmx}                  % use matching math font
\usepackage{times}                     % we use Times as the main font
\renewcommand*\ttdefault{txtt}         % a nicer typewriter font
\usepackage{cite}                      % needed to automatically sort the references
\usepackage{tabu}                      % only used for the table example
\usepackage{booktabs}                  % only used for the table example
%% We encourage the use of mathptmx for consistent usage of times font
%% throughout the proceedings. However, if you encounter conflicts
%% with other math-related packages, you may want to disable it.


%% If you are submitting a paper to a conference for review with a double
%% blind reviewing process, please replace the value ``0'' below with your
%% OnlineID. Otherwise, you may safely leave it at ``0''.
\onlineid{0}

%% declare the category of your paper, only shown in review mode
\vgtccategory{Research}

%% allow for this line if you want the electronic option to work properly
\vgtcinsertpkg

%% In preprint mode you may define your own headline.
%\preprinttext{To appear in an IEEE VGTC sponsored conference.}

%% Paper title.

\title{Augmented Reality Based Computer Science Learning}

%% This is how authors are specified in the conference style

%% Author and Affiliation (single author).
\author{Chris MacKenzie\thanks{e-mail: cmack816@colostate.edu}}
\affiliation{\scriptsize CS464}

%% Author and Affiliation (multiple authors with single affiliations).
%%\author{Roy G. Biv\thanks{e-mail: roy.g.biv@aol.com} %
%%\and Ed Grimley\thanks{e-mail:ed.grimley@aol.com} %
%%\and Martha Stewart\thanks{e-mail:martha.stewart@marthastewart.com}}
%%\affiliation{\scriptsize Martha Stewart Enterprises \\ Microsoft Research}

%% Author and Affiliation (multiple authors with multiple affiliations)
%%\author{Chris MacKenzie\thanks{e-mail: cmack816@colostate.edu}\\ %
%%        \scriptsize CS464 %


%% A teaser figure can be included as follows, but is not recommended since
%% the space is now taken up by a full width abstract.
%\teaser{
%  \includegraphics[width=1.5in]{sample.eps}
%  \caption{Lookit! Lookit!}
%}

%% Abstract section.
\abstract{Education is the cornerstone upon which civilized societies are built. For thousands of years, education has been improving using various technologies. From word-of-mouth and stone tablets to books and computers, people are always trying to find innovative new ways to pass on information to the next generation using the technology of their time. Well, technology has progressed by leaps and bounds over the last half century, and people are just now catching up antiquated educative practices to better mesh with existing technologies. In 2022, online learning and online school are already commonplace, but is this the end-all of technology in the classroom? This project is aimed to discover what new possibilities are out there with the cutting-edge world of Augmented Reality (AR) and especially how it relates to education and teaching people more effectively. %
} % end of abstract

%% ACM Computing Classification System (CCS). 
%% See <http://www.acm.org/class/1998/> for details.
%% The ``\CCScat'' command takes four arguments.


%% Copyright space is enabled by default as required by guidelines.
%% It is disabled by the 'review' option or via the following command:
% \nocopyrightspace

%%%%%%%%%%%%%%%%%%%%%%%%%%%%%%%%%%%%%%%%%%%%%%%%%%%%%%%%%%%%%%%%
%%%%%%%%%%%%%%%%%%%%%% START OF THE PAPER %%%%%%%%%%%%%%%%%%%%%%
%%%%%%%%%%%%%%%%%%%%%%%%%%%%%%%%%%%%%%%%%%%%%%%%%%%%%%%%%%%%%%%%%

\begin{document}

%% The ``\maketitle'' command must be the first command after the
%% ``\begin{document}'' command. It prepares and prints the title block.

%% the only exception to this rule is the \firstsection command
\firstsection{Introduction}

\maketitle

%% \section{Introduction} %for journal use above \firstsection{..} instead
The technology to achieve AR even for the modest consumer has been around for over a decade by now. However, very little progress has been made as far as a mainstream adoption of AR technology in the classroom is concerned. AR technology has tremendous potential for “rapid information transfer” [5] which is essentially the textbook definition of education. AR technology may also be considered more exciting by students than a typical textbook reading and so it has the possibility to boost students’ interest and interactions in a classroom. Not only this, but assignments and exercises can be made that were previously unthinkable. Think of a physics assignment that is trying to teach you about pulley systems. A textbook may have a few pictures and many paragraphs of text about these pulleys. But an augmented reality assignment could bring the pulleys to life, allowing you to really see what it looks life in real life without the need for setting up an experiment. This extremely simple example is just one of effectively infinite possibilities of using AR to better educate and engage with the world’s students. 

	This project will aim to allow students to place 3D objects using AR to assist with education. This project will contribute a free, open-source way for teachers to increase interactivity in their lesson plans. It is the hope of this project that increasing the interactivity of lesson plans through the use of AR will cause an increased amount of interest and knowledge retention in students.


\section{Related Work}

There has already been many applications and experiments concerning AR in an educative environment.  As discussed by Kesim and Ozarslan, there are many different technologies in which AR or VR can be delivered. Head-mounted displays, handheld displays, and pinch gloves are all common ways of interacting with these technologies [3]. This paper discusses the possibility of "textbooks [becoming] dynamic sources of information." [3] With the technology available, "any existing book [can be] developed into an augmented reality edition." [3] This idea to take a traditional form of information dissemination and add on to it is using AR technology is what this project is based on. 

	A study performed in 2008 by Freitas & Campos showed that AR in children's education suggested that it "is effective in maintaining high levels of motivation among children, and ... has a positive impact on the students' learning experience, especially among the weaker students." [4] This study followed 54 students between 7 and 8 years of age, and broke them into 2 groups, one who would be given a traditional lesson and one who would be given a lesson with the help of the AR technology. This study found that students' motivation was extremely high when using the AR technology, and students who were in the group that used the AR technology had a larger increase in test scores before versus after the experiment. This type of experiment is very similar to what I will be performing to determine if my application helps to increase knowledge retention and motivation.
	
	AR in education has already been applied to a wide range of topics. Topics including astronomy, chemistry, biology, mathematics, and physics have already had applications developed that incorporate AR learning into a traditional educative environment. [2] It was found that in most of these cases, AR learning provided an increase in interest into a topic by would-be learners. It is thought by Kangdon Lee that AR in education has a vast potential to engage students moreso than other traditional ways of teaching.



\section{Methodology}

	For this project, I will be performing a questionnaire after allowing 2 groups of adults to learn a concept. One group of adults will be learning a concept in a traditional fashion, through watching a lecture. Another group of adults will be learning in a more modern fashion, a lecture in conjunction with the AR application that is being developed. After each group of adults has finished, a questionnaire will be given that tests their understanding of the concept taught. This questionnaire will undergo statistical analysis using chi-square to determine if there is a significant difference in understanding between groups of adults. 
	
	
	The specific details of this user evaluation are as follows. The concept that will be taught will be how to draw control flow graphs from a specific (small) piece of pseudocode. There will be between 5 to 10 adults per group. Each group will be given 30 minutes of instruction time. The first group will only be given a lecture video to watch. The second group will be given a lecture video to watch, and the AR app in development. After 30 minutes, each group will be given a questionnaire that tests their understanding of how to draw control flow graphs. These questionnaires will also contain subjective questions about how motivated they felt, if they enjoyed the experience, and if they were paying attention or were bored. These questionnaires will then be analyzed to determine if either group performed better, and if any other conclusions can be made.



\section{Conclusion}

All in all, this project aims to serve students that are not engaged by traditional methods of teaching and show them that learning can be fun. The introduction of AR into educative environments provides a great potential for furthering lesson plans into the future. This application will be developed, and then an experiment involving 2 groups of adults will be done to determine if this app could possibly be a help to any teachers. 

%% if specified like this the section will be committed in review mode


%\bibliographystyle{abbrv}
\bibliographystyle{abbrv-doi}
%\bibliographystyle{abbrv-doi-narrow}
%\bibliographystyle{abbrv-doi-hyperref}
%\bibliographystyle{abbrv-doi-hyperref-narrow}

\bibliography{template}

[1] Chen P., Liu X., Cheng W., Huang R. (2017) A review of using Augmented Reality in Education from 2011 to 2016. In: Popescu E. et al. (eds) Innovations in Smart Learning. Lecture Notes in Educational Technology. Springer, Singapore. https://doi.org/10.1007/978-981-10-2419-1_2

[2] Lee, K. Augmented Reality in Education and Training. TECHTRENDS TECH TRENDS 56, 13–21 (2012). https://doi.org/10.1007/s11528-012-0559-3

[3] Mehmet Kesim, Yasin Ozarslan, Augmented Reality in Education: Current Technologies and the Potential for Education, Procedia - Social and Behavioral Sciences, Volume 47, 2012, Pages 297-302, ISSN 1877-0428, https://doi.org/10.1016/j.sbspro.2012.06.654.

[4] Rubina Freitas and Pedro Campos. 2008. SMART: a SysteM of Augmented Reality for Teaching 2nd grade students. In Proceedings of the 22nd British HCI Group Annual Conference on People and Computers: Culture, Creativity, Interaction - Volume 2 (BCS-HCI '08). BCS Learning & Development Ltd., Swindon, GBR, 27–30.

[5] Yuen, Steve Chi-Yin; Yaoyuneyong, Gallayanee; and Johnson, Erik (2011) "Augmented Reality: An Overview and Five Directions for AR in Education," Journal of Educational Technology Development and Exchange (JETDE): Vol. 4 : Iss. 1 , Article 11.
DOI: 10.18785/jetde.0401.10

\end{document}
